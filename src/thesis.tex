%--------------------------------------------------------------------%
%
% Berkas utama templat LaTeX.
%
% author Petra Barus, Peb Ruswono Aryan
% updated by Dionesius Agung (2020)
%--------------------------------------------------------------------%
%
% Berkas ini berisi struktur utama dokumen LaTeX yang akan dibuat.
%
%--------------------------------------------------------------------%

\documentclass[12pt, a4paper, onecolumn, oneside, final]{report}

%-------------------------------------------------------------------%
%
% Konfigurasi dokumen LaTeX untuk laporan tesis IF ITB
%
% @author Petra Novandi
% updated by Dionesius Agung (2020)
%-------------------------------------------------------------------%
%
% Berkas asli berasal dari Steven Lolong
%
%-------------------------------------------------------------------%

% Ukuran kertas
\special{papersize=210mm,297mm}

% Setting margin
\usepackage[top=3cm,bottom=3cm,left=4cm,right=3cm]{geometry}

% Uncomment this line to use TNR font instead of CMU
%\usepackage{mathptmx}

% Judul bahasa Indonesia
\usepackage[bahasa]{babel}

% Line satu setengah spasi
\renewcommand{\baselinestretch}{1.5}


%%%%%%%%%%%%%%%%%%%%%%%%%%%%%%%
%  BIBLIOGRAPHY AND CITATION  %
%%%%%%%%%%%%%%%%%%%%%%%%%%%%%%%
% use package biblatex
\usepackage[backend=bibtex,
            bibstyle=authoryear,
            citestyle=authoryear,
            sorting=nyt,
            url=false,
            maxcitenames=2,
            maxnames=2,
            dashed=false,
            giveninits=true]{biblatex}
\DeclareNameAlias{author}{last-first}

% Translate bibliography strings ke bahasa indonesia
%   (karena 'bahasa' belum di-support)
\DefineBibliographyStrings{english}{%
  bibliography = {Daftar Pustaka},
  references = {Referensi},
  and = {dan},
  techreport = {Dok. teknis},
  phdthesis = {Disertasi doktoral\adddot},
  andothers = {dkk\adddot}
}

% Field berikut tidak ditulis di daftar pustaka
\AtEveryBibitem{\clearfield{issn}}
\AtEveryBibitem{\clearfield{isbn}}
\AtEveryBibitem{\clearfield{month}}
\AtEveryBibitem{\clearfield{doi}}

% Format entri daftar pustaka

%% Beri titik setelah judul
\DeclareFieldFormat
[article,inbook,incollection,inproceedings,
patent,unpublished,misc]
{title}{#1\isdot}

%% Judul buku ditulis italic
\DeclareFieldFormat
[thesis]
{title}{\emph{#1}\isdot}

%% Hilangkan kata "Dalam:" di antara judul artikel dan judul jurnal
\renewbibmacro{in:}{}

%% Format penulisan volume dan nomor pada jurnal: vol(num) e.g. 5(1)
\renewbibmacro*{volume+number+eid}{%
	\printfield{volume}%
	\printfield{number}%
	\setunit{\addcomma\space}%
	\printfield{eid}}
\DeclareFieldFormat[article]{number}{\mkbibparens{#1}}

% Format citation
\renewcommand*{\nameyeardelim}{\addcomma\space}

% Setting spasi di halaman daftar pustaka
\setlength\bibitemsep{0.5\baselineskip}


%%%%%%%%%%%%%%
%  PACKAGES  %
%%%%%%%%%%%%%%
\usepackage[utf8]{inputenc}
\usepackage{graphicx}
\usepackage{titling}
\usepackage{blindtext}
\usepackage{sectsty}
\usepackage{chngcntr}
\usepackage{etoolbox}
\usepackage{hyperref}       % Package untuk link di daftar isi.
\usepackage{titlesec}       % Package Format judul
\usepackage{parskip}
\usepackage{booktabs}
\usepackage{tabularx}


%%%%%%%%%%%%%%$$%%%%%%%%%
%  CHAPTER AND SECTION  %
%%%%%%%%%%%%%%%%$$%%%%%%%
% Format judul bab
\chapterfont{\centering \large}
\titleformat{\chapter}[display]
  {\large\centering\bfseries}
  {\chaptertitlename\ \thechapter}{0em}
    {\large\bfseries\MakeUppercase}
\titlespacing*{\chapter}
	{0pt}
	{-1.5\baselineskip}
	{1.5\baselineskip}

% Format judul section (dan sub(sub)section)
\titleformat*{\section}{\bfseries\normalsize}
\titleformat*{\subsection}{\bfseries\normalsize}
\titleformat*{\subsubsection}{\bfseries\normalsize}
\titlespacing*{\section}{0pt}{1ex}{0pt}
\titlespacing*{\subsection}{0pt}{1ex}{0pt}
\titlespacing*{\subsubsection}{0pt}{1ex}{0pt}

% Kedalaman hierarki section (paling dalam subsubsection)
\setcounter{secnumdepth}{3}


%%%%%%%%%%%%%%%%%%%%%%%%%%%%%%%%%%%%%%%%%%%%%%%%%%
%  TABLE OF CONTENTS, LISTS OF FIGURES & TABLES  %
%%%%%%%%%%%%%%%%%%%%%%%%%%%%%%%%%%%%%%%%%%%%%%%%%%
\usepackage[titles]{tocloft}
\usepackage[titletoc]{appendix}
\usepackage{tocbibind}

% Kedalaman hierarki maksimum ToC
% (yang masuk ToC hanya sampai subsection: I.1.1.)
\setcounter{tocdepth}{2}

% Hilangkan gap antar-bab di ToC
\setlength{\cftbeforechapskip}{0pt}

% Tambah kata "BAB" sebelum nomor bab di daftar isi
% TODO: still problematic when used with list of appendices (uncomment these 4 following lines to reproduce the problem)
% \renewcommand{\cftchappresnum}{BAB~} % BAB before number in ToC
% \newlength{\mylen} % a scratch length
% \settowidth{\mylen}{\bfseries\cftchappresnum\cftchapaftersnum} % extra space
% \addtolength{\cftchapnumwidth}{\mylen} % add the extra space

% Pisah daftar lampiran dari ToC
%%%
\renewcommand{\appendixtocname}{Daftar Lampiran}

\makeatletter
\let\oldappendix\appendices

\renewcommand{\appendices}{%
  \clearpage
  % From now, everything goes to the app file and not to the toc
  \let\tf@toc\tf@app
  \addtocontents{app}{\protect\setcounter{tocdepth}{1}}
  \immediate\write\@auxout{%
    \string\let\string\tf@toc\string\tf@app^^J
  }
  \oldappendix
}%

\newcommand{\listofappendices}{%
  \begingroup
  \renewcommand{\contentsname}{\appendixtocname}
  \let\@oldstarttoc\@starttoc
  \def\@starttoc##1{\@oldstarttoc{app}}
  % Reusing the code for \tableofcontents with different
  %   \contentsname and different file handle app
  \tableofcontents
  \endgroup
}
\makeatother
%%%

% Hilangkan gap antara entri gambar & tabel antarbab di daftar tabel 
% dan daftar gambar (hanya terlihat kalau ada gambar/tabel di >1 bab)
\newcommand*{\noaddvspace}{\renewcommand*{\addvspace}[1]{}}
\addtocontents{lof}{\protect\noaddvspace}
\addtocontents{lot}{\protect\noaddvspace}


%%%%%%%%%%%%%%%%%%%%%%%%%%%%%%%%%%%%%%%%%
%  FLOATS: FIGURES, TABLES, ALGORITHMS  %
%%%%%%%%%%%%%%%%%%%%%%%%%%%%%%%%%%%%%%%%%
% Before:
% ---
% Counter untuk figure dan table.
% \counterwithin{figure}{section}
% \counterwithin{table}{section}
% ---

\usepackage[labelsep=period,
justification=justified,
format=hang]{caption}
\usepackage[labelformat=simple]{subcaption}
%% Hack subfigure cross-ref agar pakai tanda kurung
%%   e.g. Gambar II.2(a), bukan Gambar II.2a
%% (method recommended in subcaption package documentation)
\renewcommand\thesubfigure{(\alph{subfigure})}

% Counter untuk gambar dan tabel
\renewcommand*{\thefigure}{\thechapter.\arabic{figure}}
\renewcommand*{\thetable}{\thechapter.\arabic{table}}

% Jarak spasi antara float dengan teks utama
\captionsetup[figure]{belowskip=-1em}
\captionsetup[subfigure]{belowskip=0pt}
\setlength{\textfloatsep}{2\baselineskip}
\setlength{\intextsep}{2\baselineskip}

% Spasi single di environment table
\AtBeginEnvironment{table}
  {\renewcommand{\baselinestretch}{1.0}}

% Avoid widow and orphan lines if possible
\widowpenalty500
\clubpenalty10000

% Hyphenation penalty
\hyphenpenalty=10000
\tolerance=1
\sloppy


%%%%%%%%%%%%%%%%%%%%%%%%%
%  MATHS AND EQUATIONS  %
%%%%%%%%%%%%%%%%%%%%%%%%%
\usepackage{amsmath}
\usepackage{amsfonts}
\usepackage{mathtools}

% Counter untuk equation
\renewcommand*{\theequation}{\thechapter.\arabic{equation}}

% Allow page breaks on long equations
\allowdisplaybreaks[1-4]

% Operator dan notasi custom tambahan
% contoh: argmin dan argmax
\DeclareMathOperator*{\argmax}{argmax}
\DeclareMathOperator*{\argmin}{argmin}
% contoh: notasi bayes p(x | y)
\newcommand{\bayes}[2]{P(#1 \mid #2)\xspace}

\makeatletter

\makeatother

\bibliography{references}

\begin{document}

    %Basic configuration
    \title{Panduan Tugas Akhir Informatika}
    \date{\today}
    \author{
        Dionesius Agung Andika P \\
        NIM: 13516043
    }

    \pagenumbering{roman}
    \setcounter{page}{0}

    \clearpage
\pagestyle{empty}

% Setting margin for cover page
\newgeometry{top=3cm,bottom=3cm,left=3cm,right=3cm}

% Use Times font for cover page as per the thesis document guidelines
{\fontfamily{ptm}\selectfont%
\begin{center}
    
    \smallskip

    \large{\bfseries \MakeUppercase{\thetitle}}
    \\[2\baselineskip]

    \large{\bfseries Laporan Tugas Akhir}
    \\[\baselineskip]

    \normalsize{ \bfseries
    	Disusun sebagai syarat kelulusan tingkat sarjana
	}
    \\[3\baselineskip]

    \normalsize{ \bfseries Oleh\\}
    \large{ \bfseries \MakeUppercase{\theauthor}}

    \vfill
    \begin{figure}[h]
        \centering
      	\includegraphics[height=3.5cm,keepaspectratio]{resources/cover-ganesha.jpg}
    \end{figure}
    \vfill

    \large{ \bfseries
	    \uppercase{
	        Program Studi Teknik Informatika \\
	        Sekolah Teknik Elektro dan Informatika \\
	        Institut Teknologi Bandung\\
	    }
    	Juni 2020
	}

\end{center}
}%

\restoregeometry
\clearpage

    % ganti menjadi approval-single-advisor jika pembimbing 1 orang
    \clearpage
\pagestyle{empty}

\begin{center}
    \large{\bfseries \MakeUppercase{\thetitle}}
    \\[2\baselineskip]

    \large{\textbf{Laporan Tugas Akhir}}
    \\[2\baselineskip]

    \normalsize{Oleh\\
    \MakeUppercase{\textbf{\theauthor}}\\
    \textbf{Program Studi Teknik Informatika} \\
    Sekolah Teknik Elektro dan Informatika \\
    Institut Teknologi Bandung}
    \\[3\baselineskip]


    \normalsize{Telah disetujui dan disahkan sebagai Laporan Tugas Akhir\\
    di Bandung, pada tanggal \thedate}

    \vfill
    \normalsize{%
    \setlength{\tabcolsep}{12pt}
    \begin{tabular}{c@{\hskip 0.5in}c}
        Pembimbing I, & Pembimbing II, \\
        & \\
        & \\
        & \\
        & \\
        \underline{Nama dan Gelar Pembimbing I} & \underline{Nama dan Gelar Pembimbing II} \\
        NIP. 123456789 & NIP. 123456789 \\
    \end{tabular}
    }

\end{center}
\clearpage

%    \clearpage
\pagestyle{empty}

\begin{center}

    \large{\bfseries \MakeUppercase{\thetitle}}
    \\[2\baselineskip]

    \large{\textbf{Laporan Tugas Akhir}}
    \\[2\baselineskip]

    \normalsize{Oleh\\
    \MakeUppercase{\textbf{\theauthor}}\\
    \textbf{Program Studi Teknik Informatika} \\
    Sekolah Teknik Elektro dan Informatika \\
    Institut Teknologi Bandung}
    \\[3\baselineskip]


    \normalsize{Telah disetujui dan disahkan sebagai Laporan Tugas Akhir\\
    di Bandung, pada tanggal \thedate\\[3\baselineskip]
    Pembimbing,\\[4\baselineskip]
    \underline{Nama dan Gelar Pembimbing}\\
    NIP. 123456789}

\end{center}
\clearpage

    \chapter*{Lembar Pernyataan}

Dengan ini saya menyatakan bahwa:

\begin{enumerate}

    \item Pengerjaan dan penulisan Laporan Tugas Akhir ini dilakukan tanpa menggunakan bantuan yang tidak dibenarkan.
    \item Segala bentuk kutipan dan acuan terhadap tulisan orang lain yang digunakan di dalam penyusunan laporan tugas akhir ini telah dituliskan dengan baik dan benar.
    \item Laporan Tugas Akhir ini belum pernah diajukan pada program pendidikan di perguruan tinggi mana pun.

\end{enumerate}

Jika terbukti melanggar hal-hal di atas, saya bersedia dikenakan sanksi sesuai dengan Peraturan Akademik dan Kemahasiswaan Institut Teknologi Bandung bagian Penegakan Norma Akademik dan Kemahasiswaan khususnya Pasal 2.1 dan Pasal 2.2.
\\

Bandung, \thedate \\[4\baselineskip]
\underline{Dionesius Agung Andika P} \\
NIM 13516043


    \pagestyle{plain}

    % Frontmatter
    \clearpage
\chapter*{Abstrak}
\addcontentsline{toc}{chapter}{ABSTRAK}

\begin{center}
	\large{\bfseries{
			\MakeUppercase\thetitle
		}
	}
	
	\normalsize{
		Oleh\\
		\MakeUppercase \theauthor
	}
\end{center}

\medskip

\begin{spacing}{1.0}
	
	%taruh abstrak bahasa indonesia di sini
	\blindtext
	
	\blindtext
	
	Kata kunci: \LaTeX, tugas akhir, template, teknik informatika
	
\end{spacing}

\clearpage
    \clearpage
\chapter*{Abstract}
\addcontentsline{toc}{chapter}{ABSTRACT}

\begin{center}
	\large{\bfseries{
			\MakeUppercase{Computer Science Final Project Report}
		}
	}
	
	\normalsize{
		By\\
		\MakeUppercase \theauthor
	}
\end{center}

\medskip

\begin{spacing}{1.0}
	
	%taruh abstrak bahasa inggris di sini bila diperlukan
	\blindtext
	
	\blindtext
	
	Keywords: computer science, final project, \LaTeX, template
	
\end{spacing}

\clearpage
    \chapter*{Kata Pengantar}
\addcontentsline{toc}{chapter}{KATA PENGANTAR}

Gunakan bagian ini untuk memberikan ucapan terima kasih kepada semua pihak yang secara langsung atau tidak langsung membantu penyelesaian tugas akhir, termasuk pemberi beasiswa jika ada. Utamakan untuk memberikan ucapan terima kasih kepada tim pembimbing tugas akhir dan staf pengajar atau pihak program studi, bahkan sebelum mengucapkan terima kasih kepada keluarga. Ucapan terima kasih sebaiknya bukan hanya menyebutkan nama orang saja, tetapi juga memberikan penjelasan bagaimana bentuk bantuan/dukungan yang diberikan. Gunakan bahasa yang baik dan sopan serta memberikan kesan yang enak untuk dibaca. Sebagai contoh: “Tidak lupa saya ucapkan terima kasih kepada teman dekat saya, Tito, yang sejak satu tahun terakhir ini selalu memberikan semangat dan mengingatkan saya apabila lengah dalam mengerjakan Tugas Akhir ini. Tito juga banyak membantu mengoreksi format dan layout tulisan. Apresiasi saya sampaikan kepada pemberi beasiswa, Yayasan Beasiswa, yang telah memberikan bantuan dana kuliah dan biaya hidup selama dua tahun. Bantuan dana tersebut sangat membantu saya untuk dapat lebih fokus dalam menyelesaikan pendidikan saya. ....”. Ucapan permintaan maaf karena kekurangsempurnaan hasil Tugas Akhir tidak perlu ditulis.


    % Hacks to capitalize all chapter-level titles in ToC
    \renewcommand*\contentsname{DAFTAR ISI}
    \renewcommand*\appendixtocname{DAFTAR LAMPIRAN}
    \renewcommand*\listfigurename{DAFTAR GAMBAR}
    \renewcommand*\listtablename{DAFTAR TABEL}
    \renewcommand*\bibname{DAFTAR PUSTAKA}

    % Lanjutan frontmatter
    \tableofcontents
    \listofappendices
    {%
		\let\oldnumberline\numberline%
		\renewcommand{\numberline}{\figurename~\oldnumberline}%
		\listoffigures%
	}
	{%
		\let\oldnumberline\numberline%
		\renewcommand{\numberline}{\tablename~\oldnumberline}%
		\listoftables%
	}

    
    %----------------------------------------------------------------%
    % Konfigurasi Bab
    %----------------------------------------------------------------%
    %------------------------------------------------------%
    % Hack: 2 baris berikut dipindah ke chapter-1.tex
    %   previous method: nomor halaman sebelum BAB I jadi 0
    %------------------------------------------------------%
    % \pagenumbering{arabic}
    % \setcounter{page}{0}
    \renewcommand{\chaptername}{BAB}
    \renewcommand{\thechapter}{\Roman{chapter}}
    %----------------------------------------------------------------%

    %----------------------------------------------------------------%
    % Dafter Bab
    % Untuk menambahkan daftar bab, buat berkas bab misalnya `chapter-6` di direktori `chapters`, dan masukkan ke sini.
    %----------------------------------------------------------------%
    \chapter{Pendahuluan}
% Hack: gatau kenapa harus gini
\pagenumbering{arabic}
\setcounter{page}{1}

Bab Pendahuluan secara umum yang dijadikan landasan kerja dan arah kerja penulis tugas akhir, berfungsi mengantar pembaca untuk membaca laporan tugas akhir secara keseluruhan.

\section{Latar Belakang}

Latar Belakang berisi dasar pemikiran, kebutuhan atau alasan yang menjadi ide dari topik tugas akhir. Tujuan utamanya adalah untuk memberikan informasi secukupnya kepada pembaca agar memahami topik yang akan dibahas.  Saat menuliskan bagian ini, posisikan anda sebagai pembaca – apakah anda tertarik untuk terus membaca?

\section{Tujuan}

Rumusan Masalah berisi masalah utama yang dibahas dalam tugas akhir. Rumusan masalah yang baik memiliki struktur sebagai berikut:

\begin{enumerate}
    \item Penjelasan ringkas tentang kondisi/situasi yang ada sekarang terkait dengan topik utama yang dibahas Tugas Akhir.
    \item Pokok persoalan dari kondisi/situasi yang ada, dapat dilihat dari kelemahan atau kekurangannya. Bagian ini merupakan inti dari rumusan masalah.
    \item Elaborasi lebih lanjut yang menekankan pentingnya untuk menyelesaikan pokok persoalan tersebut.
    \item Usulan singkat terkait dengan solusi yang ditawarkan untuk menyelesaikan persoalan.
\end{enumerate}

Penting untuk diperhatikan bahwa persoalan yang dideskripsikan pada subbab ini akan dipertanggungjawabkan di bab Evaluasi apakah terselesaikan atau tidak.

\section{Tujuan}

Tuliskan tujuan utama dan/atau tujuan detil yang akan dicapai dalam pelaksanaan tugas akhir. Fokuskan pada hasil akhir yang ingin diperoleh setelah tugas akhir diselesaikan, terkait dengan penyelesaian persoalan pada rumusan masalah. Penting untuk diperhatikan bahwa tujuan yang dideskripsikan pada subbab ini akan dipertanggungjawabkan di akhir pelaksanaan tugas akhir apakah tercapai atau tidak.

\section{Batasan Masalah}

Tuliskan batasan-batasan yang diambil dalam pelaksanaan tugas akhir. Batasan ini dapat dihindari (tidak perlu ada) jika topik/judul tugas akhir dibuat cukup spesifik.

\section{Metodologi}

Tuliskan semua tahapan yang akan dilalui selama pelaksanaan tugas akhir. Tahapan ini spesifik untuk menyelesaikan persoalan tugas akhir. Tahapan studi literatur tidak perlu dituliskan karena ini adalah pekerjaan yang harus Anda lakukan selama proses pelaksanaan tugas akhir.

\section{Sistematika Pembahasan}

Subbab ini berisi penjelasan ringkas isi per bab. Penjelasan ditulis satu paragraf per bab buku.

    \chapter{Tinjauan Pustaka}

Bab Studi Literatur digunakan untuk mendeskripsikan kajian literatur yang terkait dengan persoalan tugas akhir. Tujuan studi literatur adalah:

\begin{enumerate}
    \item menunjukkan kepada pembaca adanya gap seperti pada rumusan masalah yang memang belum terselesaikan,
    \item memberikan pemahaman yang secukupnya kepada pembaca tentang teori atau pekerjaan terkait yang terkait langsung dengan penyelesaian persoalan, serta
    \item menyampaikan informasi apa saja yang sudah ditulis/dilaporkan oleh pihak lain (peneliti/Tugas Akhir/Tesis) tentang hasil penelitian/pekerjaan mereka yang sama atau mirip kaitannya dengan persoalan tugas akhir.
\end{enumerate}

\blindtext

\blindtext

\section{Dasar Teori}
Perujukan literatur dapat dilakukan dengan menambahkan entri baru di berkas. Tulisan ini merujuk pada \parencite{knuth2001art}

    \subsection{Bekerja dengan Float}

    Float adalah \textit{container} untuk elemen-elemen dokumen yang tidak dapat dipisah menjadi beberapa halaman. Environment ``table'' dan ``figure'' secara default adalah float. Float berguna untuk memudahkan peletakan objek yang tidak cukup jika diletakkan di halaman sekarang. Peletakan float diatur oleh \LaTeX\ dan pengguna sebaiknya memberikan keleluasaan kepada \LaTeX\ agar dapat mengatur peletakan dengan baik. 
    
    \subsubsection{Gambar}
    
    Float bisa di-\textit{cross reference}. Contohnya Gambar~\ref{fig:contoh_gambar} adalah contoh gambar.

    \begin{figure}[h]
        \centering
        \includegraphics[width=0.8\textwidth]{resources/chapter-2-infrastructure-diagram.png}
        \caption{Contoh gambar}
        \label{fig:contoh_gambar}
    \end{figure}

    \subsubsection{Tabel}

    Tabel juga merupakan float. Tabel~\ref{table:contoh_tabel} adalah contoh tabel.

    \begin{table}[htbp]
        \small
        \centering
        \caption{Contoh Tabel}
        \label{table:contoh_tabel}
        \begin{tabular}{ll}
            \toprule
            \multicolumn{1}{l}{\textbf{Contoh Judul Kolom}} & \multicolumn{1}{l}{\textbf{Nilai}}\\
            \midrule
            Besaran 1 & 12 meter          \\
            Besaran 2 & $360^\circ$       \\
            Besaran 3 & 0,2 meter         \\
            Besaran 4 & $1^\circ$         \\
            Besaran 5 & 8000 sampel/detik \\
            \bottomrule
        \end{tabular}
    \end{table}

\section{Studi Terkait}
\blindtext

    \chapter{ANALISIS DAN PERANCANGAN}

\section{Analisis Masalah}
\blindtext

\section{Solusi Umum}
\blindtext

\section{Rancangan Solusi}
\blindtext
    \chapter{EVALUASI DAN PEMBAHASAN}

\section{Tujuan Pengujian}
\blindtext

\section{Skenario Pengujian}
\blindtext

\section{Hasil Pengujian}
\blindtext

\section{Pembahasan}
\blindtext
    \chapter{PENUTUP}

\section{Kesimpulan}
\blindtext

\section{Saran}
\blindtext
    %----------------------------------------------------------------%

    % Daftar pustaka
    \begingroup
        \renewcommand{\baselinestretch}{1.0}
        \printbibliography[heading=bibintoc]
    \endgroup

    % Before:
    % ---
    % Index
    % \appendix
    % \addcontentsline{toc}{part}{Lampiran}
    % \part*{Lampiran}
    % ---

    % Format judul bab lampiran
    \titleformat{\chapter}[hang]
      {\large\bfseries}
      {\chaptertitlename\ \thechapter}{1em}
        {\large\bfseries}
    \titlespacing*{\chapter}{0pt}{-1.5\baselineskip}{\parskip}
    
    \begin{appendices}
        \input{chapters/appendix-1}
        \input{chapters/appendix-2}
    \end{appendices}

\end{document}
